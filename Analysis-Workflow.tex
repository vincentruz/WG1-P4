\documentclass[]{article}
\usepackage{lmodern}
\usepackage{amssymb,amsmath}
\usepackage{ifxetex,ifluatex}
\usepackage{fixltx2e} % provides \textsubscript
\ifnum 0\ifxetex 1\fi\ifluatex 1\fi=0 % if pdftex
  \usepackage[T1]{fontenc}
  \usepackage[utf8]{inputenc}
\else % if luatex or xelatex
  \ifxetex
    \usepackage{mathspec}
  \else
    \usepackage{fontspec}
  \fi
  \defaultfontfeatures{Ligatures=TeX,Scale=MatchLowercase}
\fi
% use upquote if available, for straight quotes in verbatim environments
\IfFileExists{upquote.sty}{\usepackage{upquote}}{}
% use microtype if available
\IfFileExists{microtype.sty}{%
\usepackage[]{microtype}
\UseMicrotypeSet[protrusion]{basicmath} % disable protrusion for tt fonts
}{}
\PassOptionsToPackage{hyphens}{url} % url is loaded by hyperref
\usepackage[unicode=true]{hyperref}
\hypersetup{
            pdftitle={SEISMIC WG1-P4},
            pdfborder={0 0 0},
            breaklinks=true}
\urlstyle{same}  % don't use monospace font for urls
\usepackage[margin=1in]{geometry}
\usepackage{color}
\usepackage{fancyvrb}
\newcommand{\VerbBar}{|}
\newcommand{\VERB}{\Verb[commandchars=\\\{\}]}
\DefineVerbatimEnvironment{Highlighting}{Verbatim}{commandchars=\\\{\}}
% Add ',fontsize=\small' for more characters per line
\usepackage{framed}
\definecolor{shadecolor}{RGB}{248,248,248}
\newenvironment{Shaded}{\begin{snugshade}}{\end{snugshade}}
\newcommand{\KeywordTok}[1]{\textcolor[rgb]{0.13,0.29,0.53}{\textbf{#1}}}
\newcommand{\DataTypeTok}[1]{\textcolor[rgb]{0.13,0.29,0.53}{#1}}
\newcommand{\DecValTok}[1]{\textcolor[rgb]{0.00,0.00,0.81}{#1}}
\newcommand{\BaseNTok}[1]{\textcolor[rgb]{0.00,0.00,0.81}{#1}}
\newcommand{\FloatTok}[1]{\textcolor[rgb]{0.00,0.00,0.81}{#1}}
\newcommand{\ConstantTok}[1]{\textcolor[rgb]{0.00,0.00,0.00}{#1}}
\newcommand{\CharTok}[1]{\textcolor[rgb]{0.31,0.60,0.02}{#1}}
\newcommand{\SpecialCharTok}[1]{\textcolor[rgb]{0.00,0.00,0.00}{#1}}
\newcommand{\StringTok}[1]{\textcolor[rgb]{0.31,0.60,0.02}{#1}}
\newcommand{\VerbatimStringTok}[1]{\textcolor[rgb]{0.31,0.60,0.02}{#1}}
\newcommand{\SpecialStringTok}[1]{\textcolor[rgb]{0.31,0.60,0.02}{#1}}
\newcommand{\ImportTok}[1]{#1}
\newcommand{\CommentTok}[1]{\textcolor[rgb]{0.56,0.35,0.01}{\textit{#1}}}
\newcommand{\DocumentationTok}[1]{\textcolor[rgb]{0.56,0.35,0.01}{\textbf{\textit{#1}}}}
\newcommand{\AnnotationTok}[1]{\textcolor[rgb]{0.56,0.35,0.01}{\textbf{\textit{#1}}}}
\newcommand{\CommentVarTok}[1]{\textcolor[rgb]{0.56,0.35,0.01}{\textbf{\textit{#1}}}}
\newcommand{\OtherTok}[1]{\textcolor[rgb]{0.56,0.35,0.01}{#1}}
\newcommand{\FunctionTok}[1]{\textcolor[rgb]{0.00,0.00,0.00}{#1}}
\newcommand{\VariableTok}[1]{\textcolor[rgb]{0.00,0.00,0.00}{#1}}
\newcommand{\ControlFlowTok}[1]{\textcolor[rgb]{0.13,0.29,0.53}{\textbf{#1}}}
\newcommand{\OperatorTok}[1]{\textcolor[rgb]{0.81,0.36,0.00}{\textbf{#1}}}
\newcommand{\BuiltInTok}[1]{#1}
\newcommand{\ExtensionTok}[1]{#1}
\newcommand{\PreprocessorTok}[1]{\textcolor[rgb]{0.56,0.35,0.01}{\textit{#1}}}
\newcommand{\AttributeTok}[1]{\textcolor[rgb]{0.77,0.63,0.00}{#1}}
\newcommand{\RegionMarkerTok}[1]{#1}
\newcommand{\InformationTok}[1]{\textcolor[rgb]{0.56,0.35,0.01}{\textbf{\textit{#1}}}}
\newcommand{\WarningTok}[1]{\textcolor[rgb]{0.56,0.35,0.01}{\textbf{\textit{#1}}}}
\newcommand{\AlertTok}[1]{\textcolor[rgb]{0.94,0.16,0.16}{#1}}
\newcommand{\ErrorTok}[1]{\textcolor[rgb]{0.64,0.00,0.00}{\textbf{#1}}}
\newcommand{\NormalTok}[1]{#1}
\usepackage{graphicx,grffile}
\makeatletter
\def\maxwidth{\ifdim\Gin@nat@width>\linewidth\linewidth\else\Gin@nat@width\fi}
\def\maxheight{\ifdim\Gin@nat@height>\textheight\textheight\else\Gin@nat@height\fi}
\makeatother
% Scale images if necessary, so that they will not overflow the page
% margins by default, and it is still possible to overwrite the defaults
% using explicit options in \includegraphics[width, height, ...]{}
\setkeys{Gin}{width=\maxwidth,height=\maxheight,keepaspectratio}
\IfFileExists{parskip.sty}{%
\usepackage{parskip}
}{% else
\setlength{\parindent}{0pt}
\setlength{\parskip}{6pt plus 2pt minus 1pt}
}
\setlength{\emergencystretch}{3em}  % prevent overfull lines
\providecommand{\tightlist}{%
  \setlength{\itemsep}{0pt}\setlength{\parskip}{0pt}}
\setcounter{secnumdepth}{0}
% Redefines (sub)paragraphs to behave more like sections
\ifx\paragraph\undefined\else
\let\oldparagraph\paragraph
\renewcommand{\paragraph}[1]{\oldparagraph{#1}\mbox{}}
\fi
\ifx\subparagraph\undefined\else
\let\oldsubparagraph\subparagraph
\renewcommand{\subparagraph}[1]{\oldsubparagraph{#1}\mbox{}}
\fi

% set default figure placement to htbp
\makeatletter
\def\fps@figure{htbp}
\makeatother

\usepackage{etoolbox}
\makeatletter
\providecommand{\subtitle}[1]{% add subtitle to \maketitle
  \apptocmd{\@title}{\par {\large #1 \par}}{}{}
}
\makeatother

\title{SEISMIC WG1-P4}
\providecommand{\subtitle}[1]{}
\subtitle{Analysis Workflow}
\author{}
\date{\vspace{-2.5em}Updated: May 19, 2020}

\begin{document}
\maketitle

{
\setcounter{tocdepth}{2}
\tableofcontents
}
\paragraph{- R/RStudio used for
analyses}\label{rrstudio-used-for-analyses}

\paragraph{- Jupyter Notebooks/Google Co-Lab used for sharing
code}\label{jupyter-notebooksgoogle-co-lab-used-for-sharing-code}

\section{\texorpdfstring{\textbf{I. Data Processing (Institution
Specific)}}{I. Data Processing (Institution Specific)}}\label{i.-data-processing-institution-specific}

\paragraph{\texorpdfstring{\emph{Note:}}{Note:}}\label{note}

\begin{itemize}
\tightlist
\item
  Exact syntax for these steps is likely to vary based on
  institution-specific variable naming conventions
\item
  See
  \href{https://docs.google.com/spreadsheets/d/1SzU4PcIEUsAGnKKyAcugHO2O2aZW29sf9a_cC-FAElk/edit\#gid=1679989021}{Data
  Description} for shared SEISMIC variable names
\item
  Sample code only shown here; see
  \href{https://github.com/seismic2020/WG1-P4}{WG1-P4 GitHub repository}
  for complete examples of institution-specific data cleaning code
\end{itemize}

\subsection{0. Startup}\label{startup}

\subsubsection{a. Load R pkgs}\label{a.-load-r-pkgs}

\begin{Shaded}
\begin{Highlighting}[]
\ControlFlowTok{if}\NormalTok{ (}\OperatorTok{!}\KeywordTok{require}\NormalTok{(}\StringTok{"pacman"}\NormalTok{)) }\KeywordTok{install.packages}\NormalTok{(}\StringTok{"pacman"}\NormalTok{)}
\KeywordTok{library}\NormalTok{(pacman)}
\NormalTok{pacman}\OperatorTok{::}\KeywordTok{p_load}\NormalTok{(}\StringTok{"tidyverse"}\NormalTok{)   }\CommentTok{# Data wrangling}

\CommentTok{# etc...}
\end{Highlighting}
\end{Shaded}

\subsubsection{b. Load full dataset}\label{b.-load-full-dataset}

\begin{Shaded}
\begin{Highlighting}[]
\CommentTok{# CHANGE TO YOUR FILE PATH}
\NormalTok{df_full <-}\StringTok{ }\KeywordTok{read.csv}\NormalTok{(}\StringTok{"~/YOUR FILE PATH HERE.csv"}\NormalTok{)}
\KeywordTok{head}\NormalTok{(}\KeywordTok{names}\NormalTok{(df_full))}
\end{Highlighting}
\end{Shaded}

\begin{verbatim}
## [1] "Example RAW Varnames"
\end{verbatim}

\begin{verbatim}
##  [1] "X"                        "EMPLID_H"                
##  [3] "ACADEMIC_PLAN_DESCR"      "ACADEMIC_PROGRAM_CD"     
##  [5] "ACADEMIC_PROGRAM_DESCR"   "BIRTH_DT"                
##  [7] "GENDER_CD"                "ETHNIC_GROUP_CD"         
##  [9] "ETHNIC_AMIND_FLG"         "ETHNIC_ASIAN_FLG"        
## [11] "ETHNIC_BLACK_FLG"         "ETHNIC_HISPA_FLG"        
## [13] "ETHNIC_PACIF_FLG"         "ETHNIC_WHITE_FLG"        
## [15] "CITIZENSHIP_STATUS_DESCR"
\end{verbatim}

\subsection{1. Clean Student level
variables}\label{clean-student-level-variables}

\subsubsection{a. Rename and generate/recode student level variables as
needed to match common SEISMIC AP variable
names}\label{a.-rename-and-generaterecode-student-level-variables-as-needed-to-match-common-seismic-ap-variable-names}

\paragraph{\texorpdfstring{\emph{Note:}}{Note:}}\label{note-1}

\begin{itemize}
\tightlist
\item
  Student level data should contain unique rows per student
\end{itemize}

\begin{Shaded}
\begin{Highlighting}[]
\NormalTok{## Student Level ####}
\NormalTok{df_std <-}\StringTok{ }\NormalTok{df_full }\OperatorTok
\StringTok{  }\CommentTok{# Renamed variables}
\StringTok{  }\KeywordTok{mutate}\NormalTok{(}\DataTypeTok{st_id =}\NormalTok{ EMPLID_H) }\OperatorTok
\StringTok{  }\KeywordTok{mutate}\NormalTok{(}\DataTypeTok{ethniccode =}\NormalTok{ ETHNIC_GROUP_CD) }\OperatorTok
\StringTok{  }\KeywordTok{mutate}\NormalTok{(}\DataTypeTok{famincome =} \KeywordTok{abs}\NormalTok{(AGI)) }\OperatorTok
\StringTok{  }\CommentTok{# Recoded variables}
\StringTok{  }\KeywordTok{mutate}\NormalTok{(}\DataTypeTok{firstgen =} \KeywordTok{recode}\NormalTok{(FIRST_GENERATION_DESCR, }\StringTok{"First Generation"}\NormalTok{ =}\StringTok{ }\DecValTok{1}\NormalTok{, }\StringTok{"Not First Generation"}\NormalTok{ =}\StringTok{ }\DecValTok{0}\NormalTok{, }\StringTok{"Unknown"}\NormalTok{ =}\StringTok{ }\DecValTok{0}\NormalTok{)) }\OperatorTok
\StringTok{  }\KeywordTok{mutate}\NormalTok{(}\DataTypeTok{ethniccode_cat =} \KeywordTok{recode}\NormalTok{(ETHNIC_GROUP_CD, }\StringTok{"HISPA"}\NormalTok{ =}\StringTok{ }\DecValTok{1}\NormalTok{, }\StringTok{"BLACK"}\NormalTok{ =}\StringTok{ }\DecValTok{1}\NormalTok{, }\StringTok{"AMIND"}\NormalTok{ =}\StringTok{ }\DecValTok{1}\NormalTok{, }\StringTok{"PACIF"}\NormalTok{ =}\StringTok{ }\DecValTok{1}\NormalTok{, }\StringTok{"ASIAN"}\NormalTok{ =}\StringTok{ }\DecValTok{2}\NormalTok{, }\StringTok{"WHITE"}\NormalTok{ =}\StringTok{ }\DecValTok{0}\NormalTok{)) }\OperatorTok
\StringTok{  }\KeywordTok{mutate}\NormalTok{(}\DataTypeTok{urm =} \KeywordTok{recode}\NormalTok{(ETHNIC_GROUP_CD, }\StringTok{"HISPA"}\NormalTok{ =}\StringTok{ }\DecValTok{1}\NormalTok{, }\StringTok{"BLACK"}\NormalTok{ =}\StringTok{ }\DecValTok{1}\NormalTok{, }\StringTok{"AMIND"}\NormalTok{ =}\StringTok{ }\DecValTok{1}\NormalTok{, }\StringTok{"PACIF"}\NormalTok{ =}\StringTok{ }\DecValTok{1}\NormalTok{, }\StringTok{"ASIAN"}\NormalTok{ =}\StringTok{ }\DecValTok{0}\NormalTok{, }\StringTok{"WHITE"}\NormalTok{ =}\StringTok{ }\DecValTok{0}\NormalTok{)) }\OperatorTok
\StringTok{  }\KeywordTok{mutate}\NormalTok{(}\DataTypeTok{gender =} \KeywordTok{recode}\NormalTok{(GENDER_CD, }\StringTok{"F"}\NormalTok{=}\DecValTok{1}\NormalTok{, }\StringTok{"M"}\NormalTok{=}\DecValTok{0}\NormalTok{, }\StringTok{"m"}\NormalTok{=}\DecValTok{0}\NormalTok{, }\StringTok{"U"}\NormalTok{ =}\StringTok{ }\DecValTok{2}\NormalTok{)) }\OperatorTok
\StringTok{  }\KeywordTok{mutate}\NormalTok{(}\DataTypeTok{female =} \KeywordTok{recode}\NormalTok{(GENDER_CD, }\StringTok{"F"}\NormalTok{=}\DecValTok{1}\NormalTok{, }\StringTok{"M"}\NormalTok{=}\DecValTok{0}\NormalTok{, }\StringTok{"m"}\NormalTok{=}\DecValTok{0}\NormalTok{, }\StringTok{"U"}\NormalTok{ =}\StringTok{ }\DecValTok{2}\NormalTok{)) }\OperatorTok
\StringTok{  }\KeywordTok{mutate}\NormalTok{(}\DataTypeTok{lowincomeflag =} \KeywordTok{if_else}\NormalTok{(}\KeywordTok{is.na}\NormalTok{(AGI), }\DecValTok{0}\NormalTok{,}
                                \KeywordTok{if_else}\NormalTok{(AGI }\OperatorTok{<=}\StringTok{ }\DecValTok{46435}\NormalTok{, }\DecValTok{1}\NormalTok{,}\DecValTok{0}\NormalTok{))) }

\CommentTok{# etc...}
\end{Highlighting}
\end{Shaded}

\begin{verbatim}
## [1] "Example Student Level Dataframe"
\end{verbatim}

\begin{verbatim}
## # A tibble: 10 x 24
##    st_id firstgen ethniccode ethniccode_cat   urm gender female famincome
##    <fct>    <dbl> <fct>               <dbl> <dbl>  <dbl>  <dbl>     <int>
##  1 1EBC~        0 WHITE                   0     0      1      1        NA
##  2 D998~        0 BLACK                   1     1      1      1     89431
##  3 7E8D~        0 WHITE                   0     0      0      0        NA
##  4 88D5~        0 WHITE                   0     0      1      1    239316
##  5 6EB5~        0 WHITE                   0     0      0      0     88705
##  6 8CE6~        0 WHITE                   0     0      1      1    144175
##  7 AEBD~        0 WHITE                   0     0      1      1     96640
##  8 FECC~        1 ASIAN                   2     0      1      1     28949
##  9 6A09~        0 WHITE                   0     0      0      0    277531
## 10 017F~        0 WHITE                   0     0      1      1    546018
## # ... with 16 more variables: lowincomeflag <dbl>, transfer <dbl>,
## #   international <dbl>, ell <dbl>, us_hs <dbl>, cohort <dbl>,
## #   cohort_2013 <dbl>, cohort_2014 <dbl>, cohort_2015 <dbl>, cohort_2016 <dbl>,
## #   cohort_2017 <dbl>, cohort_2018 <dbl>, apyear <dbl>, englsr <dbl>,
## #   mathsr <dbl>, hsgpa <dbl>
\end{verbatim}

\subsection{2. Clean Course level
variables}\label{clean-course-level-variables}

\subsubsection{a. Rename and generate/recode course level variables as
needed to match common SEISMIC AP variable
names}\label{a.-rename-and-generaterecode-course-level-variables-as-needed-to-match-common-seismic-ap-variable-names}

\paragraph{\texorpdfstring{\emph{Note:}}{Note:}}\label{note-2}

\begin{itemize}
\tightlist
\item
  Course level data will likely contain multiple rows for each course,
  per student
\end{itemize}

\begin{Shaded}
\begin{Highlighting}[]
\NormalTok{## Course Level ####}
\NormalTok{df_crs <-}\StringTok{ }\NormalTok{df_full }\OperatorTok
\StringTok{  }\CommentTok{# Renamed variables}
\StringTok{  }\KeywordTok{mutate}\NormalTok{(}\DataTypeTok{st_id =}\NormalTok{ EMPLID_H) }\OperatorTok
\StringTok{  }\KeywordTok{mutate}\NormalTok{(}\DataTypeTok{crs_sbj =}\NormalTok{ SUBJECT_CD) }\OperatorTok
\StringTok{  }\KeywordTok{mutate}\NormalTok{(}\DataTypeTok{crs_catalog =}\NormalTok{ CATALOG_NBR) }\OperatorTok
\StringTok{  }\KeywordTok{mutate}\NormalTok{(}\DataTypeTok{crs_name   =}\NormalTok{ CLASS_TITLE) }\OperatorTok
\StringTok{  }\KeywordTok{mutate}\NormalTok{(}\DataTypeTok{crs_retake =}\NormalTok{ REPEAT_CD) }\OperatorTok
\StringTok{  }\KeywordTok{mutate}\NormalTok{(}\DataTypeTok{crs_term   =}\NormalTok{ TERM_CD) }\OperatorTok
\StringTok{  }\CommentTok{# Recoded variables}
\StringTok{  }\KeywordTok{mutate}\NormalTok{(}\DataTypeTok{numgrade =}\NormalTok{ GRADE_POINTS}\OperatorTok{/}\NormalTok{UNITS_TAKEN) }\OperatorTok
\StringTok{  }\KeywordTok{mutate}\NormalTok{(}\DataTypeTok{numgrade_w =} \KeywordTok{if_else}\NormalTok{(COURSE_GRADE_CD }\OperatorTok{==}\StringTok{ "W"}\NormalTok{, }\DecValTok{1}\NormalTok{, }\DecValTok{0}\NormalTok{)) }\OperatorTok
\StringTok{  }\KeywordTok{separate}\NormalTok{(}\KeywordTok{as.character}\NormalTok{(}\StringTok{"TERM_CD"}\NormalTok{), }\KeywordTok{c}\NormalTok{(}\StringTok{"crs_YEAR"}\NormalTok{, }\StringTok{"crs_SEMESTER"}\NormalTok{), }\DecValTok{3}\NormalTok{, }\DataTypeTok{remove =} \OtherTok{FALSE}\NormalTok{) }\OperatorTok
\StringTok{  }\KeywordTok{separate}\NormalTok{(}\KeywordTok{as.character}\NormalTok{(}\StringTok{"crs_YEAR"}\NormalTok{), }\KeywordTok{c}\NormalTok{(}\StringTok{"crs_DEC"}\NormalTok{, }\StringTok{"crs_YEAR"}\NormalTok{), }\DecValTok{1}\NormalTok{) }\OperatorTok\StringTok{ }
\StringTok{  }\KeywordTok{mutate}\NormalTok{(}\DataTypeTok{crs_term_yr =}\NormalTok{ crs_YEAR) }\OperatorTok
\StringTok{  }\KeywordTok{mutate}\NormalTok{(}\DataTypeTok{crs_term_sem =}\NormalTok{ crs_SEMESTER) }\OperatorTok
\StringTok{  }\KeywordTok{mutate}\NormalTok{(}\DataTypeTok{summer_crs =} \KeywordTok{if_else}\NormalTok{(}\KeywordTok{endsWith}\NormalTok{(}\KeywordTok{as.character}\NormalTok{(TERM_CD),}\StringTok{"7"}\NormalTok{), }\DecValTok{1}\NormalTok{, }\DecValTok{0}\NormalTok{))}

\CommentTok{# etc...}
\end{Highlighting}
\end{Shaded}

\begin{verbatim}
## [1] "Example Course Level Dataframe"
\end{verbatim}

\begin{verbatim}
## # A tibble: 670,191 x 17
##    st_id crs_sbj crs_catalog crs_name numgrade numgrade_w crs_retake crs_term
##    <fct> <fct>   <fct>       <fct>       <dbl>      <dbl> <fct>         <int>
##  1 1343~ CS      0441        DISCRET~     4             0 N              2171
##  2 1343~ STAT    1000        APPLIED~     4             0 N              2174
##  3 1343~ STAT    1361        STATSTC~     3.25          0 N              2184
##  4 1343~ MATH    0240        ANALYTC~     4             0 N              2187
##  5 1343~ HPS     1616        ARTFCL ~     4             0 N              2191
##  6 1343~ STAT    1631        INTERME~     3.25          0 N              2191
##  7 AE6D~ CHEM    0310        ORGANIC~     0             0 N              2084
##  8 AE6D~ MATH    0230        ANALYTC~     0             0 N              2114
##  9 AE6D~ PHYS    0175        BASC PH~     0             0 N              2114
## 10 AE6D~ MATH    0230        ANALYTC~     0             0 N              2117
## # ... with 670,181 more rows, and 9 more variables: crs_term_yr <chr>,
## #   crs_term_sem <chr>, summer_crs <dbl>, TERM_REF <int>,
## #   enrl_from_cohort <dbl>, crs_credits <dbl>, crs_component <fct>,
## #   class_number <int>, current_major <fct>
\end{verbatim}

\subsubsection{b. For each subject course (1 and 2), create dataframe of
only first time taking that
course}\label{b.-for-each-subject-course-1-and-2-create-dataframe-of-only-first-time-taking-that-course}

\paragraph{\texorpdfstring{\emph{Note:}}{Note:}}\label{note-3}

\begin{itemize}
\tightlist
\item
  This step selects down to only a single row per course, per student
\end{itemize}

\begin{Shaded}
\begin{Highlighting}[]
\CommentTok{# By Course (Taking only First Attempt) ####}
\CommentTok{# Bio}
\NormalTok{df_crs_bio1 <-}\StringTok{ }\NormalTok{df_crs }\OperatorTok
\StringTok{  }\KeywordTok{filter}\NormalTok{(crs_sbj }\OperatorTok{==}\StringTok{ "BIOSC"} \OperatorTok{&}\StringTok{ }\NormalTok{(crs_catalog }\OperatorTok{==}\StringTok{ "0150"}\NormalTok{)) }\OperatorTok\StringTok{ }\CommentTok{# | crs_catalog == "0715")) %>%}
\StringTok{  }\CommentTok{#Only first time taking course}
\StringTok{  }\KeywordTok{group_by}\NormalTok{(st_id, crs_catalog) }\OperatorTok\StringTok{ }
\StringTok{  }\KeywordTok{arrange}\NormalTok{(crs_term, }\DataTypeTok{.by_group=} \OtherTok{TRUE}\NormalTok{) }\OperatorTok
\StringTok{  }\KeywordTok{mutate}\NormalTok{(}\DataTypeTok{crs_retake_num =} \KeywordTok{row_number}\NormalTok{()) }\OperatorTok
\StringTok{  }\KeywordTok{filter}\NormalTok{(crs_retake_num }\OperatorTok{==}\StringTok{ }\DecValTok{1}\NormalTok{) }

\CommentTok{# etc...}
\CommentTok{# repeat for Bio2, Chem1, Chem2, Phys1, Phys2}
\end{Highlighting}
\end{Shaded}

\begin{verbatim}
## [1] "Example Course Level Dataframe - First Course Only"
\end{verbatim}

\begin{verbatim}
## # A tibble: 7,368 x 18
##    st_id crs_sbj crs_catalog crs_name numgrade numgrade_w crs_retake crs_term
##    <fct> <fct>   <fct>       <fct>       <dbl>      <dbl> <fct>         <int>
##  1 0036~ BIOSC   0150        FOUNDAT~     0             0 R              2131
##  2 003E~ BIOSC   0150        FOUNDAT~     3             0 N              2101
##  3 0042~ BIOSC   0150        FOUNDAT~     2             0 N              2121
##  4 0042~ BIOSC   0150        FOUNDAT~     3.25          0 N              2071
##  5 0043~ BIOSC   0150        FOUNDAT~     3.25          0 N              2164
##  6 0043~ BIOSC   0150        FOUNDAT~     1             0 R              2121
##  7 0047~ BIOSC   0150        FOUNDAT~     3             0 N              2101
##  8 004D~ BIOSC   0150        FOUNDAT~     3             0 N              2191
##  9 0061~ BIOSC   0150        FOUNDAT~     3.75          0 N              2161
## 10 0067~ BIOSC   0150        FOUNDAT~     2             0 N              2121
## # ... with 7,358 more rows, and 10 more variables: crs_term_yr <chr>,
## #   crs_term_sem <chr>, summer_crs <dbl>, TERM_REF <int>,
## #   enrl_from_cohort <dbl>, crs_credits <dbl>, crs_component <fct>,
## #   class_number <int>, current_major <fct>, crs_retake_num <int>
\end{verbatim}

\subsection{3. Clean AP Level variables}\label{clean-ap-level-variables}

\subsubsection{a. For each AP subject, rename and generate/recode course
level variables as needed to match common SEISMIC AP variable
names}\label{a.-for-each-ap-subject-rename-and-generaterecode-course-level-variables-as-needed-to-match-common-seismic-ap-variable-names}

\paragraph{\texorpdfstring{\emph{Note:}}{Note:}}\label{note-4}

\begin{itemize}
\tightlist
\item
  Taking highest (max) AP score recieved; single row per AP exam, per
  student
\end{itemize}

\begin{Shaded}
\begin{Highlighting}[]
\NormalTok{##### AP Level ####}
\CommentTok{# By course ####}
\CommentTok{# Bio}
\NormalTok{df_ap_bio <-}\StringTok{ }\NormalTok{df_full }\OperatorTok
\StringTok{  }\KeywordTok{mutate}\NormalTok{(}\DataTypeTok{st_id =}\NormalTok{ EMPLID_H) }\OperatorTok
\StringTok{  }\KeywordTok{mutate}\NormalTok{(}\DataTypeTok{aptaker =} \KeywordTok{ifelse}\NormalTok{(}\KeywordTok{is.na}\NormalTok{(BY), }\DecValTok{0}\NormalTok{, }\DecValTok{1}\NormalTok{)) }\OperatorTok
\StringTok{  }\KeywordTok{mutate}\NormalTok{(}\DataTypeTok{eligible_to_skip =} \KeywordTok{ifelse}\NormalTok{(BY }\OperatorTok{>=}\StringTok{ }\DecValTok{4} \OperatorTok{&}\StringTok{ }\OperatorTok{!}\KeywordTok{is.na}\NormalTok{(BY), }\DecValTok{1}\NormalTok{, }\DecValTok{0}\NormalTok{)) }\OperatorTok
\StringTok{  }\KeywordTok{mutate}\NormalTok{(}\DataTypeTok{eligible_to_skip_2 =} \KeywordTok{ifelse}\NormalTok{(BY }\OperatorTok{==}\StringTok{ }\DecValTok{5} \OperatorTok{&}\StringTok{ }\OperatorTok{!}\KeywordTok{is.na}\NormalTok{(BY), }\DecValTok{1}\NormalTok{, }\DecValTok{0}\NormalTok{)) }\OperatorTok
\StringTok{  }\KeywordTok{mutate}\NormalTok{(}\DataTypeTok{tookcourse =} \KeywordTok{ifelse}\NormalTok{(}
\NormalTok{    SUBJECT_CD }\OperatorTok{==}\StringTok{ "BIOSC"} \OperatorTok{&}\StringTok{ }\NormalTok{(CATALOG_NBR }\OperatorTok{==}\StringTok{ "0150"}\NormalTok{) }\OperatorTok{&}\StringTok{ }\CommentTok{# | CATALOG_NBR == "0715") & }
\StringTok{      }\NormalTok{COURSE_GRADE_CD }\OperatorTok{!=}\StringTok{ "W"}\NormalTok{, }\DecValTok{1}\NormalTok{, }\DecValTok{0}\NormalTok{)) }\OperatorTok
\StringTok{  }\KeywordTok{mutate}\NormalTok{(}\DataTypeTok{tookcourse_2 =} \KeywordTok{ifelse}\NormalTok{(}
\NormalTok{    SUBJECT_CD }\OperatorTok{==}\StringTok{ "BIOSC"} \OperatorTok{&}\StringTok{ }\NormalTok{(CATALOG_NBR }\OperatorTok{==}\StringTok{ "0160"}\NormalTok{) }\OperatorTok{&}\StringTok{ }\CommentTok{# | CATALOG_NBR == "0716") & }
\StringTok{      }\NormalTok{COURSE_GRADE_CD }\OperatorTok{!=}\StringTok{ "W"}\NormalTok{, }\DecValTok{1}\NormalTok{, }\DecValTok{0}\NormalTok{)) }\OperatorTok
\StringTok{  }\CommentTok{#mutate(apyear = ?) %>%}
\StringTok{  }\KeywordTok{mutate}\NormalTok{(}\DataTypeTok{apscore =} \KeywordTok{as.character}\NormalTok{(BY)) }\OperatorTok
\StringTok{  }\KeywordTok{mutate}\NormalTok{(}\DataTypeTok{apscore_full =} \KeywordTok{ifelse}\NormalTok{(}\KeywordTok{is.na}\NormalTok{(BY), }\DecValTok{0}\NormalTok{, BY)) }\OperatorTok
\StringTok{  }\KeywordTok{select}\NormalTok{(st_id, aptaker}\OperatorTok{:}\NormalTok{apscore_full) }\OperatorTok
\StringTok{  }\KeywordTok{group_by}\NormalTok{(st_id) }\OperatorTok
\StringTok{  }\KeywordTok{summarize_at}\NormalTok{(}\KeywordTok{vars}\NormalTok{(}\OperatorTok{-}\KeywordTok{group_cols}\NormalTok{()),max)}

\CommentTok{# etc...}
\CommentTok{# repeat for Chem, Phys}
\end{Highlighting}
\end{Shaded}

\begin{verbatim}
## [1] "Example AP Level Dataframe - Highest Score by Exam"
\end{verbatim}

\begin{verbatim}
## # A tibble: 22,976 x 8
##    st_id aptaker eligible_to_skip eligible_to_ski~ tookcourse tookcourse_2
##    <fct>   <dbl>            <dbl>            <dbl>      <dbl>        <dbl>
##  1 0000~       0                0                0          0            0
##  2 0005~       0                0                0          0            0
##  3 0009~       0                0                0          0            0
##  4 000E~       0                0                0          0            0
##  5 000E~       0                0                0          0            0
##  6 0019~       0                0                0          0            0
##  7 001D~       1                1                1          0            0
##  8 001F~       0                0                0          0            0
##  9 001F~       0                0                0          0            0
## 10 0029~       0                0                0          0            0
## # ... with 22,966 more rows, and 2 more variables: apscore <chr>,
## #   apscore_full <dbl>
\end{verbatim}

\subsection{4. Create stacked dataset}\label{create-stacked-dataset}

\subsubsection{a. Join previously generated dataframes (Student,
Course1, Course2, and AP) for each course subject (BIO, CHEM,
PHYS)}\label{a.-join-previously-generated-dataframes-student-course1-course2-and-ap-for-each-course-subject-bio-chem-phys}

\paragraph{\texorpdfstring{\emph{Note:}}{Note:}}\label{note-5}

\begin{itemize}
\tightlist
\item
  Include new variable: ``discipline'' as flag for each subject

  \begin{itemize}
  \tightlist
  \item
    BIO
  \item
    CHEM
  \item
    PHYS
  \end{itemize}
\end{itemize}

\begin{Shaded}
\begin{Highlighting}[]
\CommentTok{# Bio (N=3090)}
\NormalTok{df_bio <-}\StringTok{ }\NormalTok{df_std }\OperatorTok
\StringTok{  }\KeywordTok{right_join}\NormalTok{(df_crs_bio2, }\DataTypeTok{by =} \StringTok{"st_id"}\NormalTok{) }\OperatorTok
\StringTok{  }\KeywordTok{full_join}\NormalTok{(df_crs_bio1, }\DataTypeTok{by =} \StringTok{"st_id"}\NormalTok{) }\OperatorTok
\StringTok{  }\KeywordTok{full_join}\NormalTok{(df_ap_bio, }\DataTypeTok{by =} \StringTok{"st_id"}\NormalTok{) }\OperatorTok
\StringTok{  }\KeywordTok{mutate}\NormalTok{(}\DataTypeTok{discipline =} \StringTok{"BIO"}\NormalTok{) }\OperatorTok
\StringTok{  }\KeywordTok{mutate}\NormalTok{(}\DataTypeTok{skipped_course =} \KeywordTok{ifelse}\NormalTok{(tookcourse }\OperatorTok{==}\StringTok{ }\DecValTok{0} \OperatorTok{&}\StringTok{ }\NormalTok{tookcourse_}\DecValTok{2} \OperatorTok{==}\StringTok{ }\DecValTok{1}\NormalTok{, }\DecValTok{1}\NormalTok{, }\DecValTok{0}\NormalTok{)) }\OperatorTok
\StringTok{  }\KeywordTok{select}\NormalTok{(discipline, st_id}\OperatorTok{:}\NormalTok{hsgpa, crs_sbj.x}\OperatorTok{:}\NormalTok{current_major.x, crs_sbj.y}\OperatorTok{:}\NormalTok{current_major.y, }
\NormalTok{         aptaker, apscore, apscore_full, eligible_to_skip, }
\NormalTok{         tookcourse, tookcourse_}\DecValTok{2}\NormalTok{, skipped_course) }

\CommentTok{# etc...}
\CommentTok{# repeat for Chem, Phys}
\end{Highlighting}
\end{Shaded}

\begin{verbatim}
## [1] "Example Stacked BIO Dataframe"
\end{verbatim}

\begin{verbatim}
## # A tibble: 22,976 x 64
##    discipline st_id firstgen ethniccode ethniccode_cat   urm gender female
##    <chr>      <fct>    <dbl> <fct>               <dbl> <dbl>  <dbl>  <dbl>
##  1 BIO        0035~        0 ASIAN                   2     0      1      1
##  2 BIO        003E~        0 WHITE                   0     0      1      1
##  3 BIO        0043~        0 WHITE                   0     0      1      1
##  4 BIO        0047~        0 WHITE                   0     0      1      1
##  5 BIO        004D~        0 WHITE                   0     0      1      1
##  6 BIO        0061~        0 WHITE                   0     0      0      0
##  7 BIO        006A~        1 HISPA                   1     1      0      0
##  8 BIO        0076~        0 WHITE                   0     0      0      0
##  9 BIO        0080~        0 WHITE                   0     0      0      0
## 10 BIO        0085~        1 ASIAN                   2     0      0      0
## # ... with 22,966 more rows, and 56 more variables: famincome <int>,
## #   lowincomeflag <dbl>, transfer <dbl>, international <dbl>, ell <dbl>,
## #   us_hs <dbl>, cohort <dbl>, cohort_2013 <dbl>, cohort_2014 <dbl>,
## #   cohort_2015 <dbl>, cohort_2016 <dbl>, cohort_2017 <dbl>, cohort_2018 <dbl>,
## #   apyear <dbl>, englsr <dbl>, mathsr <dbl>, hsgpa <dbl>, crs_sbj.x <fct>,
## #   crs_catalog.x <fct>, crs_name.x <fct>, numgrade.x <dbl>,
## #   numgrade_w.x <dbl>, crs_retake.x <fct>, crs_term.x <int>,
## #   crs_term_yr.x <chr>, crs_term_sem.x <chr>, summer_crs.x <dbl>,
## #   TERM_REF.x <int>, enrl_from_cohort.x <dbl>, crs_credits.x <dbl>,
## #   crs_component.x <fct>, class_number.x <int>, current_major.x <fct>,
## #   crs_sbj.y <fct>, crs_catalog.y <fct>, crs_name.y <fct>, numgrade.y <dbl>,
## #   numgrade_w.y <dbl>, crs_retake.y <fct>, crs_term.y <int>,
## #   crs_term_yr.y <chr>, crs_term_sem.y <chr>, summer_crs.y <dbl>,
## #   TERM_REF.y <int>, enrl_from_cohort.y <dbl>, crs_credits.y <dbl>,
## #   crs_component.y <fct>, class_number.y <int>, current_major.y <fct>,
## #   aptaker <dbl>, apscore <chr>, apscore_full <dbl>, eligible_to_skip <dbl>,
## #   tookcourse <dbl>, tookcourse_2 <dbl>, skipped_course <dbl>
\end{verbatim}

\subsubsection{\texorpdfstring{b. Stack complete dataframes for each
course subject (BIO, CHEM, PHYS), including ``discipline'' indicator
variable}{b. Stack complete dataframes for each course subject (BIO, CHEM, PHYS), including discipline indicator variable}}\label{b.-stack-complete-dataframes-for-each-course-subject-bio-chem-phys-including-discipline-indicator-variable}

\begin{Shaded}
\begin{Highlighting}[]
\CommentTok{# Stacked dataframe with Bio, Chem, Phys}
\NormalTok{df_clean <-}\StringTok{ }\KeywordTok{rbind}\NormalTok{(df_bio, df_chem, df_phys)}

\CommentTok{# etc...}
\end{Highlighting}
\end{Shaded}

\paragraph{\texorpdfstring{\emph{Note:}}{Note:}}\label{note-6}

\begin{itemize}
\tightlist
\item
  Should end up with dataset structured like this
  \href{https://docs.google.com/spreadsheets/d/1Sj5kaFNGUkBhRoOH3cIPm-97UEBZmcFkbKGjzBbKWc0/edit?usp=drive_open\&ouid=118183464940790632947}{Example
  Dataset}
\end{itemize}

\begin{verbatim}
## [1] "Example RE-CODED Varnames"
\end{verbatim}

\begin{verbatim}
##  [1] "discipline"     "st_id"          "firstgen"       "ethniccode"    
##  [5] "ethniccode_cat" "urm"            "gender"         "female"        
##  [9] "famincome"      "lowincomeflag"  "transfer"       "international" 
## [13] "ell"            "us_hs"          "cohort"
\end{verbatim}

\section{\texorpdfstring{\textbf{II. Data Analysis (Same Across
Institutions)}}{II. Data Analysis (Same Across Institutions)}}\label{ii.-data-analysis-same-across-institutions}

\paragraph{\texorpdfstring{\emph{Note:}}{Note:}}\label{note-7}

\begin{itemize}
\tightlist
\item
  Syntax for these steps should be the same for all institutions (once
  data cleaning steps above are followed)
\item
  Sample code shown here; use
  \href{https://github.com/seismic2020/WG1-P4/tree/master/Shared\%20Analysis}{Shared
  Analysis file} on WG1-P4 GitHub repository for complete analysis code
\end{itemize}

\subsection{0. Startup}\label{startup-1}

\subsubsection{a. Load R pkgs}\label{a.-load-r-pkgs-1}

\begin{Shaded}
\begin{Highlighting}[]
\ControlFlowTok{if}\NormalTok{ (}\OperatorTok{!}\KeywordTok{require}\NormalTok{(}\StringTok{"pacman"}\NormalTok{)) }\KeywordTok{install.packages}\NormalTok{(}\StringTok{"pacman"}\NormalTok{)}
\KeywordTok{library}\NormalTok{(pacman)}
\NormalTok{pacman}\OperatorTok{::}\KeywordTok{p_load}\NormalTok{(}\StringTok{"tidyverse"}\NormalTok{,   }\CommentTok{# Data wrangling}
               \StringTok{"epiDisplay"}\NormalTok{)  }\CommentTok{# Display OR for logistic regressions}

\CommentTok{# etc...}
\end{Highlighting}
\end{Shaded}

\subsubsection{b. Load clean dataset}\label{b.-load-clean-dataset}

\begin{Shaded}
\begin{Highlighting}[]
\CommentTok{# CHANGE TO YOUR FILE PATH}
\NormalTok{df_clean <-}\StringTok{ }\KeywordTok{read.csv}\NormalTok{(}\StringTok{"~/YOUR FILE PATH HERE.csv"}\NormalTok{)}
\KeywordTok{head}\NormalTok{(}\KeywordTok{names}\NormalTok{(df_clean))}
\end{Highlighting}
\end{Shaded}

\begin{verbatim}
## [1] "Example RE-CODED Varnames"
\end{verbatim}

\begin{verbatim}
##  [1] "X"              "discipline"     "st_id"          "firstgen"      
##  [5] "ethniccode"     "ethniccode_cat" "urm"            "gender"        
##  [9] "female"         "famincome"      "lowincomeflag"  "transfer"      
## [13] "transfer_cred"  "international"  "ell"
\end{verbatim}

\subsection{1. Sample selection (by RQ, for each
Course)}\label{sample-selection-by-rq-for-each-course}

\subsubsection{a. Filter for student level inclusion/exclusion
criteria}\label{a.-filter-for-student-level-inclusionexclusion-criteria}

\paragraph{\texorpdfstring{\emph{Note:}}{Note:}}\label{note-8}

\begin{itemize}
\tightlist
\item
  Use
  \href{https://github.com/seismic2020/WG1-P4/tree/master/Shared\%20Analysis}{Shared
  Analysis Syntax} or see
  \href{https://docs.google.com/spreadsheets/d/1rN8W_iz1mr7lEzBGfdTZHa45wKOSLiSF8VEpChCPsmE/edit\#gid=727713658}{Inclusion/Exclusion
  Criteria} for shared sample selection procedure
\end{itemize}

\begin{Shaded}
\begin{Highlighting}[]
\NormalTok{df_clean <-}\StringTok{ }\NormalTok{df_clean }\OperatorTok
\StringTok{  }\CommentTok{# Include}
\StringTok{  }\KeywordTok{filter}\NormalTok{(transfer }\OperatorTok{==}\StringTok{ }\DecValTok{0}\NormalTok{) }\OperatorTok
\StringTok{  }\KeywordTok{filter}\NormalTok{(tookcourse_}\DecValTok{2} \OperatorTok{==}\StringTok{ }\DecValTok{1}\NormalTok{) }\OperatorTok
\StringTok{  }\KeywordTok{filter}\NormalTok{(cohort }\OperatorTok{>=}\StringTok{ }\DecValTok{2013} \OperatorTok{&}\StringTok{ }\NormalTok{cohort }\OperatorTok{<=}\StringTok{ }\DecValTok{2018}\NormalTok{) }\OperatorTok
\StringTok{  }\CommentTok{# Exclude}
\StringTok{  }\KeywordTok{filter}\NormalTok{(international }\OperatorTok{==}\StringTok{ }\DecValTok{0}\NormalTok{)}

\CommentTok{# etc...}
\end{Highlighting}
\end{Shaded}

\subsubsection{b. Create subset dataframes for each analysis
sample}\label{b.-create-subset-dataframes-for-each-analysis-sample}

\paragraph{\texorpdfstring{\emph{Note:}}{Note:}}\label{note-9}

\begin{itemize}
\tightlist
\item
  Use
  \href{https://github.com/seismic2020/WG1-P4/tree/master/Shared\%20Analysis}{Shared
  Analysis Syntax} or see
  \href{https://docs.google.com/spreadsheets/d/1rN8W_iz1mr7lEzBGfdTZHa45wKOSLiSF8VEpChCPsmE/edit\#gid=129222174}{Sample
  Descriptions} for full model specifications
\end{itemize}

\begin{Shaded}
\begin{Highlighting}[]
\CommentTok{# Bio}
\CommentTok{# Took 2nd course in sequence}
\NormalTok{df_bio2 <-}\StringTok{ }\NormalTok{df_clean }\OperatorTok
\StringTok{  }\KeywordTok{subset}\NormalTok{(discipline }\OperatorTok{==}\StringTok{ "BIO"}\NormalTok{) }\OperatorTok
\StringTok{  }\KeywordTok{subset}\NormalTok{(apyear }\OperatorTok{>=}\StringTok{ }\DecValTok{2013}\NormalTok{)}

\CommentTok{# etc...}
\CommentTok{# repeat for Chem, Phys}
\end{Highlighting}
\end{Shaded}

\begin{Shaded}
\begin{Highlighting}[]
\CommentTok{# Took AP }
\NormalTok{df_BYtakers <-}\StringTok{ }\NormalTok{df_clean }\OperatorTok
\StringTok{  }\KeywordTok{subset}\NormalTok{(discipline }\OperatorTok{==}\StringTok{ "BIO"}\NormalTok{) }\OperatorTok
\StringTok{  }\KeywordTok{subset}\NormalTok{(aptaker }\OperatorTok{==}\StringTok{ }\DecValTok{1}\NormalTok{)}

\CommentTok{# etc...}
\CommentTok{# repeat for Chem, Phys}
\end{Highlighting}
\end{Shaded}

\begin{Shaded}
\begin{Highlighting}[]
\CommentTok{# Skip eligible}
\NormalTok{df_BYeligible <-}\StringTok{ }\NormalTok{df_clean }\OperatorTok
\StringTok{  }\KeywordTok{subset}\NormalTok{(discipline }\OperatorTok{==}\StringTok{ "BIO"}\NormalTok{) }\OperatorTok
\StringTok{  }\KeywordTok{subset}\NormalTok{(eligible_to_skip }\OperatorTok{==}\StringTok{ }\DecValTok{1}\NormalTok{)}

\CommentTok{# etc...}
\CommentTok{# repeat for Chem, Phys}
\end{Highlighting}
\end{Shaded}

\begin{Shaded}
\begin{Highlighting}[]
\CommentTok{# Skip eligible, at each eligble score (}\AlertTok{NOTE}\CommentTok{: THESE WILL DIFFER BY INSTITUTION!)}
\NormalTok{df_BYeligible.}\DecValTok{4}\NormalTok{ <-}\StringTok{ }\NormalTok{df_bio2 }\OperatorTok
\StringTok{  }\KeywordTok{subset}\NormalTok{(apscore }\OperatorTok{==}\StringTok{ }\DecValTok{4}\NormalTok{)}
\NormalTok{df_BYeligible.}\DecValTok{5}\NormalTok{ <-}\StringTok{ }\NormalTok{df_bio2 }\OperatorTok
\StringTok{  }\KeywordTok{subset}\NormalTok{(apscore }\OperatorTok{==}\StringTok{ }\DecValTok{5}\NormalTok{)}

\CommentTok{# etc...}
\CommentTok{# Repeat for Chem, Phys}
\end{Highlighting}
\end{Shaded}

\subsection{2. Run Models (for each
Course)}\label{run-models-for-each-course}

\paragraph{\texorpdfstring{\emph{Note:}}{Note:}}\label{note-10}

\begin{itemize}
\tightlist
\item
  Use
  \href{https://github.com/seismic2020/WG1-P4/tree/master/Shared\%20Analysis}{Shared
  Analysis Syntax} or see
  \href{https://docs.google.com/spreadsheets/d/1rN8W_iz1mr7lEzBGfdTZHa45wKOSLiSF8VEpChCPsmE/edit\#gid=129222174}{Sample
  Descriptions} for full model specifications
\end{itemize}

\subsubsection{RQ1: What student characteristics are associated with
student participation and success in AP courses for students enrolled at
the selected
universities?}\label{rq1-what-student-characteristics-are-associated-with-student-participation-and-success-in-ap-courses-for-students-enrolled-at-the-selected-universities}

\subsubsection{\texorpdfstring{\emph{RQ1a: Who takes AP
exams?}}{RQ1a: Who takes AP exams?}}\label{rq1a-who-takes-ap-exams}

\begin{Shaded}
\begin{Highlighting}[]
\CommentTok{# Model 1a: Credits ####}
\CommentTok{#Bio}
\NormalTok{m1.a_bio <-}\StringTok{ }\KeywordTok{glm}\NormalTok{(aptaker }\OperatorTok{~}\StringTok{ }\KeywordTok{factor}\NormalTok{(firstgen) }\OperatorTok{+}\StringTok{ }\KeywordTok{factor}\NormalTok{(lowincomeflag)  }\OperatorTok{+}\StringTok{ }\KeywordTok{factor}\NormalTok{(female) }\OperatorTok{+}\StringTok{ }\KeywordTok{factor}\NormalTok{(urm) }\OperatorTok{+}
\StringTok{                 }\KeywordTok{scale}\NormalTok{(hsgpa) }\OperatorTok{+}\StringTok{ }\KeywordTok{scale}\NormalTok{(mathsr) }\OperatorTok{+}\StringTok{ }\KeywordTok{scale}\NormalTok{(englsr) }\OperatorTok{+}\StringTok{ }\KeywordTok{factor}\NormalTok{(cohort),}
               \KeywordTok{binomial}\NormalTok{(}\DataTypeTok{link =} \StringTok{"logit"}\NormalTok{), df_bio2)}
\KeywordTok{logistic.display}\NormalTok{(m1.a_bio)}

\CommentTok{# etc...}
\CommentTok{# Repeat for Chem, Phys}
\end{Highlighting}
\end{Shaded}

\begin{verbatim}
## [1] "Example: Model Parameters for RQ1a - Bio"
\end{verbatim}

\begin{verbatim}
##  
##                               OR lower95ci upper95ci     Pr(>|Z|)
## factor(firstgen)1      1.0839026 0.8285885  1.417887 5.565976e-01
## factor(lowincomeflag)1 0.9257913 0.7290026  1.175701 5.271245e-01
## factor(female)1        0.8636020 0.7363766  1.012808 7.131839e-02
## factor(urm)1           0.9646356 0.7417146  1.254555 7.882842e-01
## scale(hsgpa)           0.9837005 0.9087208  1.064867 6.845533e-01
## scale(mathsr)          1.2179055 1.1133348  1.332298 1.678186e-05
## scale(englsr)          1.0259311 0.9393439  1.120500 5.693163e-01
## factor(cohort)2014     1.1887481 0.9177678  1.539738 1.902389e-01
## factor(cohort)2015     1.3856350 1.0709123  1.792849 1.309643e-02
## factor(cohort)2016     1.3837122 1.0708700  1.787948 1.300699e-02
## factor(cohort)2017     1.7817311 1.3726503  2.312727 1.425023e-05
## factor(cohort)2018     2.5481683 1.9140613  3.392348 1.485664e-10
\end{verbatim}

\subsubsection{\texorpdfstring{\emph{RQ1b: Who gets what score on AP
exams?}}{RQ1b: Who gets what score on AP exams?}}\label{rq1b-who-gets-what-score-on-ap-exams}

\begin{Shaded}
\begin{Highlighting}[]
\CommentTok{#Bio}
\NormalTok{m1.b_bio <-}\StringTok{ }\KeywordTok{lm}\NormalTok{(}\KeywordTok{scale}\NormalTok{(apscore) }\OperatorTok{~}\StringTok{ }\KeywordTok{factor}\NormalTok{(firstgen) }\OperatorTok{+}\StringTok{ }\KeywordTok{factor}\NormalTok{(lowincomeflag)  }\OperatorTok{+}\StringTok{ }\KeywordTok{factor}\NormalTok{(female) }\OperatorTok{+}\StringTok{ }\KeywordTok{factor}\NormalTok{(urm) }\OperatorTok{+}
\StringTok{                }\KeywordTok{scale}\NormalTok{(hsgpa) }\OperatorTok{+}\StringTok{ }\KeywordTok{scale}\NormalTok{(mathsr) }\OperatorTok{+}\StringTok{ }\KeywordTok{scale}\NormalTok{(englsr) }\OperatorTok{+}\StringTok{ }\KeywordTok{factor}\NormalTok{(crs_term),}
\NormalTok{              df_BYtakers)}
\KeywordTok{summary}\NormalTok{(m1.b_bio)}

\CommentTok{# etc...}
\end{Highlighting}
\end{Shaded}

\begin{verbatim}
## [1] "Example: Model Parameters for RQ1b - Bio"
\end{verbatim}

\begin{verbatim}
##  
##                               OR lower95ci upper95ci     Pr(>|Z|)
## factor(firstgen)1      1.0445422 0.8669833 1.2584653 6.467208e-01
## factor(lowincomeflag)1 0.8774990 0.7426503 1.0368333 1.250205e-01
## factor(female)1        0.7815049 0.7017255 0.8703546 7.874963e-06
## factor(urm)1           0.7084578 0.5934068 0.8458152 1.445518e-04
## scale(hsgpa)           1.1339354 1.0712224 1.2003198 1.609665e-05
## scale(mathsr)          1.2288801 1.1572551 1.3049382 2.635409e-11
## scale(englsr)          1.3051871 1.2292475 1.3858180 9.581728e-18
## factor(crs_term)2014   0.7680279 0.6358375 0.9277006 6.256180e-03
## factor(crs_term)2015   0.7889513 0.6531683 0.9529614 1.403018e-02
## factor(crs_term)2016   0.6750136 0.5608454 0.8124225 3.438879e-05
## factor(crs_term)2017   0.7171274 0.5970589 0.8613417 3.898370e-04
## factor(crs_term)2018   0.8390635 0.6962436 1.0111800 6.553994e-02
\end{verbatim}

\subsubsection{\texorpdfstring{\emph{(Etc\ldots{})}}{(Etc\ldots{})}}\label{etc}

\subsection{3. Run Propensity Matching Models (for each Course) - COMING
SOON!}\label{run-propensity-matching-models-for-each-course---coming-soon}

\paragraph{\texorpdfstring{\emph{Note:}}{Note:}}\label{note-11}

\begin{itemize}
\tightlist
\item
  We are using inverse probability weight (IPW). The idea is to assign
  greater weights for those less likely to receive treatment (i.e.,
  skipping a course). This is because in the presence of confounding,
  estimate for the average treated and control outcomes may be biased.
\item
  More details about propensity score matching methods can be found
  here: \url{https://www.ncbi.nlm.nih.gov/pmc/articles/PMC3144483/}
\item
  A demonstration of WeightIt for IPW:
  \url{https://cran.r-project.org/web/packages/WeightIt/vignettes/WeightIt_A0_basic_use.html}
\end{itemize}

\section{\texorpdfstring{\textbf{III. Data Visualization (Same Across
Institutions) - COMING
SOON!}}{III. Data Visualization (Same Across Institutions) - COMING SOON!}}\label{iii.-data-visualization-same-across-institutions---coming-soon}

\end{document}
